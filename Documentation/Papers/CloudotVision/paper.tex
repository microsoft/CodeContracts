\documentclass{llncs}

\title{A Case for Static Analyzers in the Cloud\\ {(Position paper)}}

\author{Michael Barnett \inst{1} Mehdi Bouaziz\inst{2} Manuel F\"ahndrich \inst{1} Francesco Logozzo \inst{1}}

\institute{Microsoft Research, Redmond, WA (USA) \and \'Ecole normale sup\'erieure, Paris (France)}

%TODO [review1]
%- can you make the link with bytecode a bit earlier (e.g., say in the introduction that Clousot can verify (also) bytecode)
%TODO [review2]
%There are some challenges in implementing a massively parallel analyzer, but these challenges are little discussed, related work is not discussed (and I doubt that no analyzer has been targeting multi-core architectures)
% If there were more discussion on the related work or on the challenges, it would also be interesting
%TODO [tball]
%Do you have some numbers yet on the effect of the internal parallelization on speedup and precision with respect to the sequential Clousot?

\begin{document}
\maketitle

\begin{abstract}
We describe our ongoing effort of moving a desktop static
analyzer, Clousot, into a cloud-based one, Cloudot.
A cloud-based static analyzer runs as a service.  Clients issue analysis
requests through the local network or over the internet.  The analysis
takes advantage of the large computation resources offered by the
cloud: the underlying infrastructure ensures scaling and virtually unlimited
storage.  Cloud-based analyzers may relax performance-precision
trade-offs usually associated with desktop-based analyzers.  More
cores enable more precise and responsive analyses.  More storage
enables full caching of the analysis results, shareable among
different clients, and queryable off-line.  To realize these advantages,
cloud-based analyzers need to be architected differently than
desktop ones.
\end{abstract}

%TODO: [mehdi] we assume that all static analyzers are based on abstract interpretation, maybe we should make it clear that it is only our case and that it implies some things

\section{Desktop-based static analyzers}
\label{sec:desktop}

Traditionally, the design of static analyzers focuses on desktop users.
It  is therefore strongly influenced by the performance trade-offs made in order  to run smoothly in such a setting.

A (classical) desktop static analyzer is a batch process.  It takes as
input a program or a library, either in source or in binary form, it
analyzes it, and it reports possible errors to the user.  The analysis
phase, the most important phase of the whole process, is usually
performed sequentially on the local machine.  Once the analysis is
done, the intermediate computation is discarded, so that a new
analysis of the same program, or that of a slightly modified version,
would start from scratch.  The analysis process is quite expensive,
and on realistic code, it can vary from a few minutes to several hours.
A re-analysis from scratch significantly reduces the usability of an
analyzer when applied to large code bases.  Furthermore, the analysis
effort is not shared among a team of software engineers working on the
same code base, even though most of the code and analysis options will be identical.
Each run of the analyzer is isolated, no information nor computation
is shared.

A way to solve those problems is to run the analysis on a centralized
server: desktop users send a request to the server.  The server is
responsible for processing the analysis requests.  It can store the
results for future reuse.  It keeps a database with all the inputs and
the respective outcomes: when a client requests the analysis of a
program already in the database, the server can simply replay the
outcome, without analyzing it from scratch.

A problem with the centralized server solution is scalability: in the
case of medium or large teams of developers, it may be the case that
too many analysis requests are issued, causing lag in the response, or
overloading the server.  Adding more computing power to the server
(more cores, memory) may solve the problem; more clients can be served
in parallel, assuming each analysis requires one core. A centralized
server has the drawback that there is no elasticity in allocating
resources---most of the server resources may be unutilized, waiting
for peaks of requests.  Furthermore, each organization would have to
maintain its own server hardware, the service itself, software and
hardware updates, \emph{etc}.---a significant cost. On top of this, no sharing can be
achieved among different organizations, \emph{e.g.}, in the case of
projects jointly developed by multiple organizations.

\section{Static analyzers as Services in the Cloud}

An even better solution than a centralized server is to let the
analyzer run on top of a cloud infrastructure, as SaaS (Software
as a Service).  The cloud infrastructure provides extremely large
computational resources---large numbers of cores and huge amounts of
storage.  The service can be configured to be elastic with respect to
incoming analysis requests: the more requests, the more allocated resources.
When requests subside, the analyzer releases the extra resources.

Hosting a cloud based analysis service means that there is only one
version of the static analyzer.  This simplifies support and bug %TODO [tball] one version of the static analyzer -> the cloud could also host multiple analyzers & pick the best one, or race them
fixing. Tool developers do not need to support different versions of
the analyzer, in different configurations, on different platforms.
Furthermore, every bug fix in the analyzer is immediately and
transparently available to all the users. On the other hand, deploying
and testing newer version of the analysis service does require careful
staging and maintenance of at least the production and the next-gen
version of the service.

Different service requests, from different users, share the results of
the analysis.  Suppose that the underlying static analyzer has
functionality to identify previously analyzed code and to reuse the
analysis results (\emph{e.g.}, the inferred abstract states).  Now,
for each service request, the analyzer can save the intermediate steps
of the computation---assuming the cloud provides virtually unlimited amounts
of storage.  While serving a request, if the analyzer detects that a
piece of code has already been analyzed in a previous request, it
can simply reuse that result.  Therefore, the cloud enables an
aggressive and transparent sharing of the analysis among projects and
teams.

\section{Perspectives}

The cloud offers new and exciting research challenges and
opportunities.  It provides a (relatively) cheap massive distributed
environment.  The design of the static analyzer should take into
account the large number of available cores.  In particular, we expect
the analysis to be faster and more precise. It can be faster because
the computation can be parallelized on many nodes. The analysis can
also be more precise because the extra resources enable the
analysis to use more CPU and memory intensive algorithms than are
practical on the desktop.

But how do we structure the static analyzer to fully exploit the cloud
infrastructure?  For instance, how can we split and parallelize the
computation for a global program analysis to make it scalable to very
large programs?  In Sect.~\ref{sec:cloudot} we discuss our current
design for Cloudot, our version of the abstract interpretation-based .NET static analyzer to run in
the cloud.  We designed new distributed and highly asynchronous
fixpoint algorithms for static analysis.  Chaotic asynchronous
iterations~\cite{Cousot77} are the way to go.
Designed in the 70's, theses methods for solving fixpoint systems of
monotonic equations on multiprocessor machines are proven to converge
to the same result as iterative methods.  However, we must
explore what happens when the analysis functions are not monotonic,
\emph{e.g.}, this is the (common) case when using infinite abstract
domains with widening or finite domains with $k$-limiting.  For
non-monotonic functions, the analysis may never reach a fixpoint,
\emph{e.g.}, it may end up alternating between two incomparable
abstract values, leading to two different warnings,
which is not an acceptable situation for an
end-user. How to deal with those problems is an open research problem.

We expect data collection and mining to be one of the main advantages
of cloud-based static analyzers.  We can store all the analyzed
programs, with all the analysis options and the intermediate analysis
results.  A first immediate benefit is the ability to collect data on the effective
usage of the tool: how often and how many clients use the tool, which
options or abstract domains are most/least used, which warnings get
fixed, which ones are simply suppressed or ignored, \emph{etc}.  By
exploring the warnings, we can infer the most common precision
weaknesses of the analyzer, and then either refine the analyzer or
design new abstract domains tailored to the warnings.  In summary, a
static analyzer in the cloud will enable data-driven design of static
analyses and abstract domains.  In our opinion, this will be a giant
step enabling a better and deeper understanding of the usage and
performance of static analysis tools in the wild.

We can leverage the analysis results on previous versions of a program
to perform semantic-guided warning suppression.  The analyzer can
learn from false warnings to reduce the noise in future versions of
the same code base.  False warnings are marked by the user, for
instance using a suppress warnings mechanism.  Ideally, the analyzer
will be able to adapt to the coding style of the code base, learning code
patterns used in such a code base, and hence automatically reducing
the false warnings ratio.

The static analyzer can provide new, semantic metrics on the code to
help manage the quality of large projects.  It can report how the
ratio of warnings per line of code changes from version to version, the
evolution over time of the warning ratio for a particular piece of
code, which module in the code base has most/least warnings, \emph{etc}.  It
can combine the data with testing and concrete bug reports to figure
out the less stable parts program components.

The user can query the stored data, in order to use it, \emph{e.g.}, for code
reviews.  She can ask complex semantic queries on the program such as
which check-in introduced a warning, which caller passed a negative
argument to this method, what happens if this variable is in this range,
\emph{etc}.  The cloud can cache all such queries, so as to optimize 
their execution and share their result.


\section{Clients}
The simplest client for the cloud static analyzer simply uploads a
program (the source or a compiled binary).  Then it issues the
analysis request to the cloud service.  When the analysis is done, the
client reads back the results and shows them to the user, as a text
file, or squiggles in the editor.  As the bulk of the analyzer is in
the cloud, users do not need to install, configure, or update the
analyzer.  A tiny client suffices.

A more refined client may exploit the massive parallelism of the
cloud.  It can request the parallel analysis of the same program but
with different analysis options.  The options can specify different
abstract domains (more or less precise) for the heap and numerical
values but also different treatments of method calls (fully modular,
summary based, fully inline), join points (merge or not), \emph{etc}.
The user will first see the results from the most imprecise (usually the first
to be completed) analysis requests, and then these results will be
refined over time with more precise (but more time intensive) ones.
The main advantage is to obtain very quick but coarse analysis feedback, and
improve the precision over time.

The client can be integrated in the project build system.  When a new
version of the code base is checked in, the build system issues a
request to the cloud analyzer and the results are incorporated in the
build log.

A hybrid client can partition the computation between the local
desktop and the remote cloud.  In an interactive setting, the local
client only analyzes the code visible on the screen, while the rest of
the project is analyzed in the cloud.  As a consequence, the user will
get immediate feedback for ``visible'' code, whereas the cloud
analyzes the ``invisible'' code. As the user scrolls to other parts,
the newly visible code analysis may already be in the cache and
immediately displayed.

As an alternative, the client can also be a rich Web application.  The
programmer edits the code in a browser application.  The code is
automatically saved on the cloud and analyzed in the background by the
cloud static analyzer.  The user will see not only the warnings in
real-time but also possible fixes for them~\cite{LogozzoBall12}.

\section{Other Considerations}
Static analysis can be easily modularized on annotated code, where annotations
typically take the form of pre- and post-conditions on methods and
object invariants on types. For code under development, it is best if
these so-called contracts are directly written in the code itself,
making them available during analysis and readable to
programmers. However, often, programmers use 3rd party components that
lack contracts and a system of out-of-band contracts to annotate these
components is necessary. In a cloud setting, these out-of-band
contracts can be more easily distributed, so that the analyzer always
has the most up-to-date version of the contracts. Furthermore, a
centralized place for these out-of-band contracts makes it possible to
crowd-source them---a big advantage, as writing and maintaining them
is very time consuming. We envision a client tool to visualize and
update the out-of-band contracts from anywhere.

\section{Cloudot}
\label{sec:cloudot}
We are implementing the vision outlined above in Cloudot---Clousot in
the cloud.  Our starting point is the code contract static checker
cccheck/Clousot~\cite{MafLogozzo10}.

Clousot is a static analyzer for .NET based on abstract
interpretation~\cite{CousotCousot77}.  Clousot analyzes methods in
isolation.  For each method \texttt{m} in the program, it assumes its
precondition, and it asserts the postcondition.  For each method
\texttt{n} called from \texttt{m}, it asserts the precondition of \texttt{n}
and it assumes its postcondition.  The output is an instrumented
method \texttt{m'}.  Clousot has three phases.  First, it analyzes the
instrumented \texttt{m'} to infer invariants for each program point.
Second, it uses this information to prove assertions, either
user-provided (\emph{e.g.}, preconditions) or language-induced
(\emph{e.g.}, division by zero).  Third, it generates verified
repairs, a precondition~\cite{CousotCousotMafLogozzo13} and a
postcondition to be propagated to the callers, and an object invariant
to be used in other type members~\cite{BouazizLogozzoMaf12}.  Clousot
uses a SQL database (local or remote) to cache the results of the analysis.  It uses the
hash of the instrumented \texttt{m'} as a key into the database.  If it
finds an entry with such a key, then it simply reads the associated
data, \emph{i.e.}, the outputs of the  phases two and three above.

Our first step to turn Clousot into Cloudot was to make it able to run
as a standalone service.  Services should be fully functional: they
take a request, process it, and answer it.  The answer should only
depend on the input, not the history of the serviced requests.
Cloudot accepts as input a binary and a set of options, describing
among other things the abstract domains for the analysis and the
reference assemblies containing the contracts.  The output is a list  %TODO [tball] ??
of warnings and code repairs (including inferred contracts).  We
modified Clousot to run as a service, essentially turning all the
static variables into instance variables.  Furthermore, we worked
out the (complex) details to make the service run remotely.
Eventually, this gives us a first level of parallelism: we serve each
request independently from all the others, \emph{i.e.}, we run each
analysis request on a different core.  Internally, we cache the
assembly analysis by using a shared database.  In practice this first
version of Cloudot corresponds to the scenario of a centralized analysis server
sketched in Sect.~\ref{sec:desktop}.

Next, we wanted to make Cloudot internally parallel, so as to achieve
massive parallelism: ideally, we would assign a core to each method
analysis, and perform a global fixpoint computation to propagate
information among methods.  Clousot is a (rather complex) sequential
analyzer.  Converting it to make it parallel would be a strenuous
engineering task.  Therefore, we chose a different solution.
When Cloudot gets a request, it slices the binary into minimal % TODO [review1] - can you give an example of a typical MAU
analyzable units (MAUs).  MAUs are smaller, self-contained binaries
containing at the minimum a single method along with all metadata and
contracts of program elements referenced by that method.  Unlike executable
slices, MAUs do not need to include other methods' body.  Cloudot pushes MAUs
into a shared queue.  On the other side of the queue, workers
(essentially Clousot services) pull work from the queue, perform the
analysis, write the results into a database, and notify the completion
of the MAU analysis.  Cloudot checks if the analysis of a MAU has
reached a fixpoint, \emph{i.e.}, its output is stable.  If it is not
the case, Cloudot loads the dependencies of the MAU (essentially the
callers and the callees of methods in the MAU, and the other methods in the type)
and it pushes them again in the queue for analysis.  Cloudot iterates
until all the MAUs are stable.  We are still working on assuring
that the computation of the global fixpoint always converges.

%\section{Related Work}

% http://swreflections.blogspot.fr/2010/08/has-static-analysis-reached-its-limits.html
% http://findbugs.sourceforge.net/findbugs2.html#cloud

%http://blog.optimyth.com/2012/11/new-trend-for-2013-move-static-code-analysis-to-the-cloud
%http://searchcloudsecurity.techtarget.com/news/2240163542/The-pros-and-cons-of-cloud-based-static-code-analysis-tools

%http://www.infoq.com/articles/chess-secureprog
% https://www.fortifymyapp.com

%http://www.electric-cloud.com/blog/2012/11/19/static-code-analysis-acceleration/

%http://en.wikipedia.org/wiki/Kalistick#cite_note-16

%TODO [review3]
% The Parfait project at Oracle Labs already explored some of the issues surrounding the scaling of static analysis to large codebases and the use of static program analysis in large programming organizations and the need to centrally store and collate analysis results [see 1,2,3].
% Some of the commercial code analysis tools which have their origins in academia, like Fortify and Coverity, have also addressed this area.
% The major novelty in the approach appears to be the use of cloud-style infrastructure rather than traditional big iron servers - the paper does not clearly outline why cloud infrastructure as opposed to traditional large servers from the perspective of the analyses - use of large servers to analyze real-world programs totalling millions of lines of code requires highly scalable and parallelizable analyses.


%TODO [review3]
%[1] http://ieeexplore.ieee.org/xpl/articleDetails.jsp?arnumber=6171154
%Transitioning Parfait into a Development Tool - IEEE Security & Privacy 2012
%[2] http://dl.acm.org/citation.cfm?id=1394505
%Parfait: Designing a scalable bug checker - SAW 2008
%[3] http://dl.acm.org/citation.cfm?id=2025183
%Static deep error checking in large system applications using parfait - FSE 2011 Tool demonstration


%Mehdi:
% Parfait
%http://labs.oracle.com/projects/parfait/parfait.php
%http://labs.oracle.com/projects/downunder/publications/saw08.pdf
% C/C++ runtime errors
% demand-driven analysis (backward slicing)
% constant propagation, folding, partial evaluation, symbolic analysis using affine constraints, taint analysis
% analyses are sound and unsound
% 2 axes of parallelization: (1) analyses (because they are independent?), (2) point of interest (demand-driven)

% We can also mention Polyspace which uses a client/server approach (but that's it?)
% http://www.mathworks.com/products/polyspaceserverc/index.html

%[review1]- can you discuss how much work it would be to do a similar thing to other static analysis tools?
%Sending a static analysis tool to the cloud mainly comes down to:
%(1) making it parallel on a single machine;
%(2) making it a service, even if the interface is very simple, i.e., an everlasting process waiting for queries;
%(3) optionally, using a centralized database for results and caching;
%(4) building the cloud service machinery: service workers, waiting queues, job schedulers;
%(5) finding an axis of parallelization, with a medium granularity, e.g., some kind of slicing, or independent analyses;
%(6) depending on the analysis, global iterations may be needed to compute fixpoints.

%Not all tools can be easily parallelized in the cloud, if, e.g., they have a single non-modular analysis and programs are analyzed as a whole. %e.g. Astr�e

% What is new is not a static analysis tool in the cloud but an abstract-interpretation static analyzer in the cloud
% AND using the cloud to do chaotic iterations

\section{Conclusions}
We advocate the development of static analysis tools in the cloud.
The cloud provides huge computational resources that can be used to
improve the performance and precision of analysis tools.  The designer
of a cloud-based analyzer is free to relax most of the usual
precision-performance trade-offs adopted for desktop-based static
analyzers.  On the other hand, the analyzer must be re-architected
to exploit the model of computation of the cloud infrastructure.  We
need to devise new algorithms, for instance for asynchronous fixpoint
computation of non-monotonic functions, for the sharing of
computation, and for  load balancing between local and remote
computation.  We have started the exploration of these issues in our
ongoing effort on Cloudot.

\paragraph{Acknowledgements}
The authors wish to acknowledge Tom Ball and the anonymous reviewers for their helpful comments on the manuscript.

\bibliographystyle{plain}
\bibliography{bib}

\end{document}
